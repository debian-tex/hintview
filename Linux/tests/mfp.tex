\let\MFmanual=\!

\input manmac.tex

\pagewidth=\hsize \pageheight=\vsize \def\onepageout#1{\shipout\vbox{ % here we define one page of output
    \offinterlineskip % butt the boxes together
%    \vbox to 3pc{ % this part goes on top of the 44pc pages
%      \iftitle % the next is used for title pages
%        \global\titlefalse % reset the titlepage switch
%        \setcornerrules % for camera alignment
%      \else\ifodd\pageno \rightheadline\else\leftheadline\fi\fi
%      \vfill} % this completes the \vbox to 3pc
    \vbox to \pageheight{
      \ifvoid\margin\else % marginal info is present
        \rlap{\kern31pc\vbox to\z@{\kern4pt\box\margin \vss}}\fi
      #1 % now insert the main information
      \ifvoid\footins\else % footnote info is present
        \vskip\skip\footins \kern-3pt
        \hrule height\ruleht width\pagewidth \kern-\ruleht \kern3pt
        \unvbox\footins\fi
      \boxmaxdepth=\maxdepth
      } % this completes the \vbox to \pageheight
    }
  \advancepageno}

\proofmodefalse

\ifproofmode\message{Proof mode is on!}\fi
\titlepage
\def\rhead{Preface}
\vbox to 8pc{
\rightline{\titlefont Preface}\vss}
%{\topskip 9pc % this makes equal sinkage throughout the Preface
\vskip-\parskip
\tenpoint
\noindent\hang\hangafter-2
\smash{\lower12pt\hbox to 0pt{\hskip-\hangindent\cmman G\hfill}}\hskip-16pt
{\sc ENERATION} {\sc OF} {\sc LETTERFORMS} \strut by mathematical means
was first tried in the fifteenth century; it became popular in the
sixteenth and seventeenth centuries; and it was abandoned (for good
reasons) during the eighteenth century. Perhaps the twentieth century
will turn out to be the right time for this idea to make a comeback,
now that mathematics has advanced and computers are able to
do the calculations.

%} % end of the special \topskip

Modern printing equipment based on raster lines---in which metal ``type''
has been replaced by purely combinatorial patterns of zeroes and ones
that specify the desired position of ink in a discrete way---makes
mathematics and computer science increasingly relevant to printing.
We now have the ability to give a completely precise definition of letter
shapes that will produce essentially equivalent results on all raster-based
machines. Moreover, the shapes can be defined in terms of variable
parameters; computers can ``draw'' new fonts of characters
in seconds, making it possible for designers to perform valuable experiments
that were previously unthinkable.

\MF\ is a system for the design of alphabets suited to raster-based
devices that print or display text. The characters that you are reading
were all designed with \MF\!, in a completely precise way; and they
were developed rather hastily by the author of the system, who is a rank
amateur at such things. It seems clear that further work with \MF\ has
the potential of producing typefaces of real ^{beauty}. This manual has
been written for people who would like to help advance the art of
mathematical type design.

\vfil
\end
